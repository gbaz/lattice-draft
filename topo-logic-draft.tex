\documentclass[hoptionsi,review,format=acmsmall]{acmart}
%\geometry{landscape}                		% Activate for rotated page geometry
%\usepackage[parfill]{parskip}    		% Activate to begin paragraphs with an empty line rather than an indent
\usepackage{graphicx}				% Use pdf, png, jpg, or eps§ with pdflatex; use eps in DVI mode
								% TeX will automatically convert eps --> pdf in pdflatex
\usepackage{amssymb}

\newtheorem*{remark}{Remark}


% \newtheorem*{corollary}{Corollary}


\newtheorem{theorem}{Theorem}[section]
\newtheorem{corollary}{Corollary}[theorem]
\newtheorem{lemma}[theorem]{Lemma}

\theoremstyle{definition}
\newtheorem{definition}{Definition}[section]

%SetFonts

%SetFonts

%------------------------------------------------
%            Symbols in "mathcal"
%------------------------------------------------

\newcommand{\Mcc}{\mathcal{M}}
\newcommand{\UMc}{\mathcal{UM}}
\newcommand{\Wc}{\mathcal{W}}
\newcommand{\Oc}{\mathcal{O}}
\newcommand{\Pc}{\mathcal{P}}
\newcommand{\Uc}{\mathcal{U}}
\newcommand{\Jc}{\mathcal{J}}
\newcommand{\Cc}{\mathcal{C}}
\newcommand{\Lc}{\mathcal{L}}
\newcommand{\Dc}{\mathcal{D}}
\newcommand{\BLc}{\mathcal{BL}}

\newcommand{\band}{\mathbin{\&}}
\newcommand{\bor}{\mathbin{|}}
\newcommand{\Dia}{\diamondsuit}


\title[Concurrency and Dependency via Distributive Lattices]{The Topological and Logical Structure of Concurrency and Dependency via Distributive Lattices}
\author{Gershom Bazerman}
 \affiliation{%
   \institution{Awake Security}}
 \email{gershomb@gmail.com}
\author{Raymond Puzio}
\affiliation{%
   \institution{Albert Einstein Institute}}
\email{rspuzio@gmail.com}

\begin{abstract}
It is widely recognized as desirable to give a mathematical semantics to specifications of concurrent computations. But the purpose of a semantics is not just to exist, but to be useful in further analyzing the properties of the programs they provide models of. So it is desirable in particular to give a semantics that has familiar logical properties to facilitate reasoning, but also tractable topological properties, so that the weight of modern mathematical analysis may be brought to bear on teasing out important structural aspects. This paper makes a number of related contributions in this regard. First, it relates the specification of branching dependency structures, which exist in fields from knowledge-representation to package management, to the specification of semantics of concurrent computation. Second, it introduces a new order-theoretic construction -- the extension of the Bruns-Lakser completion to posets. It then makes use of this as a key ingredient in associating to dependency structures associated locales -- equipped with both topological and logical properties. It then provides an example of how this interplay of properties can be of use -- using topological properties of the dependency structure to equip internal logics of associated locales with a modality representing contraction relations. Finally, it discusses how such constructions may relate to important questions in complexity theory regarding SAT-solvers. Along the way, we will see how this approach relates to familiar objects such as package version policies, Merkle-trees, the nix operating system, and distributed version control tooling like git.
\end{abstract}

\begin{document}
\maketitle

\section{Introduction}
This project began with seeking to understand the mathematical structure and logic of software package repositories. Such repositories contain tens of thousands of packages, with complex webs of interlinking dependencies, represented as expressions in propositional logic, containing not only branching choices, but also a notion of "compatibility ranges" and a notion of conflict. The structures we developed for modeling this turned out to be extremely similar to work on the semantics of concurrent computation -- and for good reason! The problem of branching dependency specification is exactly the problem of concurrent computation, just "turned on its head." Semantics of concurrent computation, at base, consist of a collection of states, and certain allowable transitions between them, which may be simultaneous, and which may be nondeterministic. A dependency specification, such as given by a package repository, also has a collection of states (the collection of installed packages), and also has a collection of allowable transitions (one may only install a new package when all dependencies are satisfied). Furthermore, concurrency takes the form of distinguishing when two independent packages may be built simultaneously. And finally, in both cases, we have a notion of "incompatibility" -- the former, in terms of contention for a shared resource, and the latter, in terms of e.g. disallowing that two linked packages expose the same required symbol with different definitions. The difference is then largely in the questions asked about such structures. With concurrent semantics, the whole structure is the "program" and the typical questions asked are how such things compose. With dependency structures, a "program" is what we think of as a "build plan" -- a single trace through the structure to a particular end state, and the questions asked are about optimality, reachability, etc. Our approach is inspired by the latter way of thinking, but we think it sheds light on many related issues as well. In particular, we see that "inside" any single concurrent program, we can examine not only its state space, but also a related internal logic -- the logic of dependency specifications. This lends itself to fine-grained intensional analysis, exploiting the interplay.

In section 2, we introduce the basic elements and tools of our analysis -- dependency structures with choice, and their related "trace" structures, reachable dependency posets. Section 3 introduces the extended Bruns-Lakser completion, which is an independent order-theoretical result of general mathematical interest. We then discuss how reachable dependency posets, through the lens of Bruns-Lakser, produce a very general formal mathematical notion of what a "Merkle" structure is. Section 4 discusses two intuitionistic logics that can be built over this completion -- one an immediate "logic of paths", and one passing through the free distributive lattice over a poset to yield a "logic of requirements." Section 5 relates the topological and logical aspects of this investigation by examining how certain "covering" relations in dependency structures give rise to modalities in the associated logics discussed. Section 6 presents some preliminary investigations on how this approach may shed light on the innate topological structure of SAT problems, with regards to the difficulty they present to SAT solvers. We then conclude with a discussion on related and future work.

\section{Dependency Structures with Choice}
We begin our analysis with a definition of dependency structures. These are intended to correspond almost immediately to the data provided by package repositories -- events (packages) which may depend on a choice of other events. These structures do not (yet) have any notion of a choice of versions -- so version 1.0 and 1.1 of the same package are logically two entirely different events. Later we will see how to recover this data.

\begin{definition}
A \textbf{Pre-Dependency Structure with Choice} is a pair \((E, D : E \rightarrow \Pc(\Pc(E)))\) where \(E\) is a finite set of events, and \(D\) is a non-nullary mapping from \(E\) to its double powerset, to be interpreted as mapping each event to a set of alternative dependency requirements -- i.e. to a predicate in disjunctive normal form ranging over variables drawn from \(E\).
\end{definition}

\begin{definition}
A \textbf{Dependency Structure with Choice} (DSC) is a pre-DSC with \(D\) satisfying  appropriate conditions of transitive closure and cycle-freeness. We define \(X\) as a \textbf{possible dependency set} of \(e\) if \(X \in D(e)\). We call an event set \(X\) a \textbf{complete event sent} if for every element \(e\) there is a possible dependency set \(Y\) of \(e\) such that \(Y \subseteq X\). A pre-DSC is a DSC if every possible dependency set of every element is complete, and no possible dependency set of any element contains the element itself. Pre-DSCs may be completed into DSCs by repeatedly taking transitive completion of possible dependency sets (with regards to each transitive possible dependency set) and then deleting cyclic sets (and elements whose possible dependency set becomes empty under this process) until a fixpoint is reached.
\end{definition}

DSCs are richer than the standard notion of a dependency tree or dependency graph. In such structures, a node \(a\) with edges to \(b\) and \(c\) exhibits a dependency on both \(b\) and \(c\). There is no way, however, to express a dependency on either \(b\) or \(c\). A domain-theoretic account of such structures is given by \textit{pomsets}, introduced by Vaughn Pratt. Pomsets are a special instance of a broader class of structures, known as \textit{event structures}, introduced by Nielson, Plotkin and Winskel [CITE], and used in the domain-theoretic semantics of concurrent computation and concurrent games. Such structures have not only a (choice-free, transitively normalized) dependency relation, but in addition a conflict structure which indicates incompatible collections of events (typically presented as a collection of consistent sets which carves out only compatible collections of events). Finally, there are so-called \textit{general event structures}, which extend event structures with a notion of choice in roughly the same fashion as DSCs (i.e. by moving from a partial ordering relation into a relation between elements and powersets). Their theory is less well behaved and understood, and its study is an area of ongoing work. DSCs may be characterized as general event structures which are conflict-free. Thus intuitively (i.e. not necessarily formally), in a lattice of expressive power, DSCs sit above pomsets, "side by side" with event structures, and below general event structures. One hope of the present is work is that it might be usefully extended in some fashion with conflicts, leading among other things to a further understanding of general event structures.\footnote{When first introduced, what we now know as "general event structures" were simply named "event structures", and what we now know as "event structures" were named "stable event structures." We follow the modern convention in this paper.}

\subsection{Reachable Dependencies Posets of a DSC}

Data as given in a DSC is purely declarative. To introduce an analysis of dynamics, we need a structure which we can trace through time. From DSCs we derive a partially ordered set of execution traces, analogous to the family of configurations of an event structure.

\begin{definition} The \textbf{reachable dependency poset} of a DSC is the result of an operation, \(rdp\), which sends DSCs to bounded posets (i.e. posets with top and bottom elements) by the following two-step procedure:  We take as elements all collections of events, i.e. \(\Pc(E)\), and impose the least order relation such that one collection of events, \(X\), is above another, \(Y\), if \(Y \subset X\) and for every element of \(X\), there is a possible dependency set contained in \(Y\).  The conditions of transitivity and cycle-freeness ensure that under such an ordering, every event will lie above the empty event set. We refer to event sets that exist in this poset as \textbf{reachable event sets}.
\end{definition}

Informally, a reachable dependency set is generated by asking "for each event, what are the basic (reachable) event sets which contain it," and then completing those by the empty set and all unions of this basis. Viewed as a graph, nodes of a reachable dependency poset correspond to complete event sets, and edges correspond to linear accretion of event sets over time by addition of subsequent events. Consequently, a reachable dependency poset may also be seen as generated by considering all possible dependency sets of all events, augmenting each with the event itself, and then, under the inclusion ordering, augmenting the result with the empty set and in addition all possible joins. We leave it to the reader to convince themselves that both procedures yield the same result.

It follows that for any DSC \((E,D)\), \(rdp((E,D))\)  is a subposet of \(\Pc(E)\), and has all joins (i.e. unions). However, importantly, it does not have all meets (intersections). Consider a DSC containing an event \(a\), which depends on either \(b\) or \(c\). \(\{a,b\}\) and  \(\{a,c\}\) are reachable event sets, but their intersection, \(\{a\}\), is not. Hence we have a bounded join-poset, but not a lattice, much less a distributive lattice. As a syntactic convenience, when we denote operations whose domain is a poset on a DSC, we implicitly pass through the reachable dependency poset construction.

\section{Distributive Lattices and the Idempotent Distributive Lattice Completion}

We review here some basic facts and notation regarding order theory and lattices.

A \textbf{partially ordered set} or poset, \(P\) is a set equipped with a partial order relation \(\le\), which is transitive,  reflexive, and antisymmetric (i.e. for which \( a \le b \band b \le a \implies a = b\)). A (homo)morphism of posets is an monotone (order-preserving) function on their elements, and with such morphisms posets form the category \(Pos\) and finite posets form the subcategory \(FinPos\). Two posets are equivalent when there exist morphisms \(f, g\) between them such that \(f \odot g = id\) and \(g \odot f = id\), i.e. when they are equivalent as objects of \(Pos\). We note that all posets have a standard partial order on them such that \(P \le Q\) when there exists an order-preserving embedding \(P \rightarrow Q\).

A \textbf{lattice}, \(L\) is partially ordered set for which every two elements have a unique greatest lower bound, their \textbf{meet} (\(\wedge\)) and a unique least upper bound, their \textbf{join} (\(\vee\)). The join and meet operations of a lattice are necessarily commutative, associative, and idempotent. A (homo)morphism of distributive lattices is a morphism of posets which also preserves meets and joins. A \textbf{join-semilattice} and \textbf{meet-semilattice} are posets that respectively have all finite joins or all finite meets. A \textbf{complete lattice} is a lattice which has joins and meets of infinitary as well as finitary collections of elements. We write \(\bigwedge\) and \(\bigvee\) for the meet and join operations as applied to an entire set of elements. By abuse of notation, we also may write, e.g., \(x \vee S\) where \(x\) is an element of a poset and \(S\) is a set of elements, to indicate the lifting of application of the unary operation \(x \vee -\) to every element in the set.

A \textbf{distributive lattice}, is a lattice satisfying the additional property that for all \(x, y, z\) in \(L\), \(x \vee (y \wedge z) = (x \vee y) \wedge (x \vee z)\). It is easy to verify that if this condition (join distributing over meet) is satisfied, then the dual condition (meet distributing over join) is also satisfied.  Lattice homomorphisms between distributive lattices are necessarily distributive lattice homomorphisms, and with such morphisms distributive lattices form the category \(DLat\) and finite distributive lattices form the subcategory \(FinDLat\). In \(FinDLat\), all lattices necessarily have a unique top and bottom element (i.e. are bounded). As such, we require morphisms in \(FinDLat\) to also preserve top and bottom elements as the nullary join and meet (i.e. to be homomorphisms of bounded lattices).

A \textbf{join-irreducible} element of a poset is an element \(x\) such that no collection of elements not including \(x\) has \(x\) as its join. The operation \(\Jc(P)\) sends a poset (or a lattice viewed as a poset) to the sub-poset of its join-irreducible elements, sharing the same order relation. An intuition that this lends itself to is that join-irreducible elements are ideals. We refer to elements of a poset which are not join-irreducible as \textbf{composite} elements, and the set of join-irreducible elements which joins to them as their \textbf{basis}. It is important to note that if a poset has a globally least element (i.e. element which stands below all other elements in the order relation), that element is not join-irreducible, since it is the join of the empty set. However, if a poset has more than one locally least element (i.e. element with no element below it), then all such elements are join-irreducible. It is also important to note that even if an element is join-irreducible in \(P\), it still may nonetheless become a join in the restriction to \(\Jc(P)\).

A \textbf{downset} of a poset is a set of elements of the poset which is downwardly-closed -- i.e. for which \(x \in S \band y \le x \implies y \in S\). The operation \(\Oc(P)\) sends a poset to the poset of its downsets, ordered by inclusion. Such a poset has meets and joins as respectively intersection and union, and consequently is a distributive lattice. Further, \(\Oc(P)\) is a morphism (and in fact an embedding) of posets, which sends each \(x \in P\) to the set \(\{y \mathbin{|} y \le x\}\). The dual operation to taking downsets is taking \textbf{up-sets}  which are upwardly-closed. We denote this as \(\Uc(P)\).

A \textbf{Heyting algebra} is a lattice with a unique top and bottom element, and a special "implication" operation called the \textbf{relative pseudo-complement} (\(a \rightarrow b\)) which yields the unique greatest element \(x\) such that \(a \wedge x \le b\). A \textbf{complete Heyting algebra} is a Heyting algebra such that it is also a complete lattice. The category of complete Heyting algebras takes as morphisms monotone functions which preserve finite meets, arbitrary joins, and implication.

A \textbf{frame} is a complete Heyting algebra. However, the category \(Frm\) of frames takes as morphisms monotone functions which preserve finite meets and arbitrary joins, but not necessarily implication. This is to say that the relative pseudo-complement operation derived from finite meets and arbitrary joins necessarily exists in frames, but may not commute with any given frame homomorphism. In the finitary case, distributive lattices and complete Heyting algebras coincide, and hence \(FinFrm = FinDLat\).

A \textbf{locale} is again the same thing as a frame. However, in the category \(Loc\) of locales, morphisms are viewed reversed, and hence \(Loc = Frm^{op}\) and \(FinLoc = FinFrm^{op} = FinDLat^{op}\)

\subsection{Distributive Lattices in Logic and Topology}

Distributive lattices play a very special role in both logic and topology. From a logical standpoint, Heyting algebras provide a complete semantics for intuitionistic logic, with true and false corresponding to top and bottom, and corresponding to meets, or corresponding to joins, and implication being given by the relative pseudo-complement. More precisely, a formula in intuitionistic propositional logic is provable (tautological) if and only if it is valid (yields true under every assignment of variables) in every Heyting algebra (and in fact this result holds even when one considers only finite Heyting algebras). Hence, a finite Heyting algebra (resp. distributive lattice) is a setting where we can directly interpret expressions in intuitionistic logic. [CITE VAN DALEN]

From a topological standpoint, frames are the object of study of locale theory, i.e. so-called "pointless topology." From any topological space -- given as a set of points, and a covering relation of open sets -- the open sets themselves form an order theoretic structure which is precisely a frame (which is known, when the homomorphisms between frames are viewed backwards, as a "locale"). If we then forget the points, and consider only the frame, we can still "do" topology -- and from any frame (resp. locale) we can recover a special type of space, known as a sober space. In fact, frames and sober spaces are in one-to-one correspondence. [CITE JOHNSTONE, VICKERS]

As we observed above, our reachable dependency posets are not lattices, much less distributive. So they are not settings in which we can perform logical or topological analysis directly. However, intuitively, it feels like they \textit{should} be such a setting. In particular, events resemble points, and reachable event sets very much resemble open covers. This motivates the construction of the "best" way to derive an associated distributive lattice from a reachable dependency poset (and in fact, from any poset whatsoever).

\subsection{Bruns-Lakser for Posets}

In 1970 [CITE], Bruns and Lakser introduced an indempotent distributive lattice completion for meet-semilattices. Only later (we believe in 2014 [CITE BALL]), was it realized that this completion (though expressed in different terms) was actually introduced by  Holbrook MacNeille in the same 1937 [CITE] work where he first introduced the famed Dedekind-MacNeille completion of partially ordered sets into complete lattices. Here we introduce an extension of this result that covers all posets, as well as a novel description of its' universal property, and a simple algorithmic characterization of the completion in the finite case. Proofs of the central statements here are given in the appendix.

First, we recall the original construction of Bruns-Lakser and MacNeille.

\begin{definition}
(Bruns-Lakser) An \textbf{admissible set} is a subset \(S\) of a meet-semilattice \(P\) in which for all \(x\), \(x \wedge \bigvee(S) = \bigvee(x \wedge S)\).
\end{definition}

\begin{theorem}
(Bruns-Lakser, MacNeille) The partially ordered set of all admissible sets of a meet-semilattice \(P\) is a distributive lattice, \(\BLc(P)\). There exists an injection \(bl : P \rightarrow \BLc(P)\), which preserves all meets and joins of admissible sets. Furthermore, any morphism to a distributive lattice that preserves all meets and joins of admissible sets, \(f : P \rightarrow D\), factors uniquely into the injection \(bl : P \rightarrow \BLc(P)\) followed by a distributive lattice homomorphism \(g : \BLc(P) \rightarrow D\).
\end{theorem}

Now, we present our extension of this construction:

\begin{definition}
(Bazerman, Puzio) A set is \textbf{level-closed} when it is downward closed and closed under binary joins such that at least one element is not the suprema of at least one maximal chain containing it within the poset (i.e. where at least one element is has a higher element within the poset). The \textbf{level-closure} \(lc\) of a set is its completion with regards to the level-closed property, and preserves maxima. An \textbf{join-maximal set} is a subset \(S\) of a poset \(P\) which has a maxima and for which there is no element \(x \notin lc(S)\) such that \(\bigvee(\{x\} \cup S) = \bigvee(S) \band x \neq \bigvee(S)\). A \textbf{maximal join} is the join of a join-maximal set.
\end{definition}

\begin{theorem}
(Bazerman, Puzio) The partially ordered set of all join-maximal sets of a poset \(P\) is a distributive lattice, \(\Mcc(P)\). There exists an injection \(m : P \rightarrow \Mcc(P)\), which preserves all meets and maximal joins. Furthermore, any morphism to a distributive lattice that preserves all meets and maximal joins, \(f : P \rightarrow D\), factors uniquely into the injection \(m : P \rightarrow \Mcc(P)\) followed by a distributive lattice homomorphism \(g : \Mcc(P) \rightarrow D\). Furthermore, when \(P\) is a meet-semilattice, this construction coincides with \(\BLc\).
\end{theorem}

In both cases, it follows from the central theorem that the completion is idempotent. Further, it is functorial. \(\Mcc\) acts on morphisms that preserve meets and maximal joins by lifting their action on all elements in the original source and target, and extending their action on new elements in the source by sending them to the join in the target of the targets of their join-irreducible basis in the source. This functor in fact gives the category of distributive lattices as a reflective category of either the subcategory of meet-semilattices which shares all objects but only has morphisms that preserve meets and admissible joins, or the subcategory of posets which shares all objects but only has morphisms that preserve meets and maximal joins. [CITE GHERKE]

The set of admissible (resp. join-maximal) sets is rather large and unwieldy. But in the finite case, we have a much nicer characterization of the completion, which is computationally simple and also suggestive and familiar with regards to structures that occur elsewhere in computer science. The following is a simplification and extension of MacNeille's characterization of the finite elements of his completion:

\begin{theorem}
(Bazerman, Puzio) For a finite poset, \(\Mcc(P)\) may be constructed as \(\Oc(\Jc(P)))\), with an injection that sends join-irreducible elements to their downsets, and composite elements to the union of their join-irreducible basis.
\end{theorem}

\subsection{Merkle Structures and the Completion of Reachable Dependency Posets}

To the vaguely formed question "how do we make reachable dependency posets topological" we can now propose a precise answer: application of the idempotent distributive lattice completion. This is to say, we take the composite \(\Mcc(rdp(E,D))\) (which, by notational shorthand, may be written as \(\Mcc(E,D)\). The irreducible elements of a reachable dependency poset are those sets which are generated by the possible dependency sets of individual events -- i.e. that have an event which is shared by no complete dependency set below them. In the example where \(a\) depends on either \(b\) or \(c\), the irreducible elements, i.e. \(\Jc(rdp(E,D))\), are \(\{b\}\), \(\{c\}\), \(\{a,b\}\), and \(\{a,c\}\). Hence the application of \(\Oc\) yields the four sets \(\{\{b\}\}\), \(\{\{c\}\}\), \(\{\{b\},\{a,b\}\}\), and \(\{\{c\},\{a,c\}\}\), but also \(\{\}\), \(\{\{b\},\{c\},\{b,c\}\}\) and \(\{\{b\},\{c\},\{a,b\},\{a,c\}\}\). (see figure n.)

%TODO add illustration

In the case of finite posets presented as a collection of sets, with ordering given by inclusion (as in the case of reachable dependency posets), we can give a simple algorithmic characterization of this construction. For every event which has multiple "paths" to enable it (i.e. multiple possible dependency sets), split the event into new events, each labeled by a different possible dependency set. And since branched events may depend on other branched events, do so recursively. In the resulting structure, rather than sets of events, there are sets of events each labeled by the "path" taken to get to them. 

From the standpoint of concurrent semantics, this amounts to replacing sets of events by sets of execution traces. From the standpoint of dependency specification, this amounts to augmenting packages by their "build plan". This operation makes intuitive sense in that it provides a more granular and correct specification of what a package "really" means. For example library \(a\) may link against library \(b\) or \(c\), each of which provide the same API-surface, but which have subtle differences in behavior. So while \(a\) depends "equally" on either \(b\) or \(c\), the resulting products, \(a_b\) and \(a_c\) are not guaranteed to be the same thing. It is precisely this distinction which is captured by taking the idempotent distributive lattice completion.

The subscript "shorthand" used above renders each event as subscripted by the path of dependencies to reach it. Since that path itself consists of events, in a nontrivial chain, then those events too are subscripted, and soforth. Representing this computationally seems a bit of a chore. If we took each label and turned it into a hash, and then when taking sets of labels instead took hashes of their hashes, etc, then (with high probability) we could represent the same information in constant space rather than space geometric in the height of our poset.

In fact, this operation -- augmenting a structure such that each node contains a hashed description of the path to reach it -- is relatively ubiquitous in modern software. It lies at the heart of the model of patches in distributed version control systems such as git, and is also used in blockchains. In those cases, while meets exist, joins do not. And further, in the blockchain case, it is even simpler because the goal is to restrict "truth" to a chosen linear path -- i.e. to avoid branching or "splits" in the chain. This construction is also the basic insight of the nix operating-system, as well as the "nix-like" store used by the Cabal build system for Haskell. In these cases, meets and joins both are used. The operation of generating the path information and taking a hash-chain of it is known as producing a \textbf{Merkle tree}. [CITE MERKEL] The idempotent distributive lattice completion, \(\Mcc\) is then, in a sense, the \textit{non-probabilistic Merkle transformation} of a poset, and provides a formal description of what it means when we take an existing data structure and "turn it into a Merkle tree."

It would be remiss of us not to mention, as well, that the operation of sending elements of a poset to their downsets, a key step in this transformation, is a special case of the Yoneda embedding. In particular, it is the Yoneda embedding for (0,1)-presheaves -- i.e. when \(C^{op} \rightarrow Set\) is decategorified to \(P^{op} \rightarrow Bool\). Categorically-inclined readers may bear this in mind the next time somebody asks them what practical purpose such constructions serve.

%TODO get right functor arrow (notation)??

\section{DSCs and Finite Locales}

The above suggests an extension to our understanding of DSCs. We have seen how a DSC may be transformed into a finite locale via the extended Bruns-Lakser completion of its reachable dependency poset. We also can observe that this locale is a sublocale of the locale freely generated by the downsets of our DSC. Furthermore, that locale is, in turn, a sublocale of that freely generated by the powersets of an event-set under the inclusion ordering. Thus, it is natural to ask what general relationship exists between DSCs and finite locales. The first problem that presents itself is that even up to renaming of events, distinct DSCs can nonetheless present order-equivalent frames. For example, the DSC with \(a\) depending on \(b\) or \(c\) gives the same frame as the DSC with \(a\) depending on \(b\) and \(d\) depending on \(c\) (since the Bruns-Lakser completion "splits" the former \(a\) into two copies to begin with). However, this is exactly the same situation that occurs in topos theory (and its decategorification in locale theory) when multiple sites (categories with a topology, resp. posets with a coverage) can present the same topos (resp. locale). This suggests that our correspondence should not be to locales directly, but rather to their presentations as sites (known, for locales, as "posites") [CITE SPIVAK].

The central tool in this section is the nucleus, which is the localic analogue of a topology. We recall its definition and some basic facts regarding it. Readers are again referred to [VICKERS, JOHNSTONE] for further discussion.

A \textbf{nucleus} is the algebraic structure on a frame that gives a sublocale. It given as a monotone function on a frame \(j : L \rightarrow L\), satisfying three properties. First: \(j(a \wedge b) = j(a) \wedge j(b)\). Second, \(a \le j(a)\). Finally, \(j(ja) \le j(a)\). These may be summarized as saying that it is meet-preserving, contractive (in this case, inflationary) and idempotent. An element of \(L\) is said to be \(j\)-closed if \(j(a)=a\). Further, if \(L\) is a frame, then \(L/j\), which consists only of \(j\)-closed elements, is also a frame, and there exists a surjective frame homomorphism \(j^* : L \rightarrow L/j\). If we view a frame as a category, a nucleus is just a left-exact idempotent monad; if we view a frame as a decategorified topos, a nucleus is a topology; and if we view a frame as generated by its internal logic, a nucleus gives a "possibility" or "locally true" modal operator on that logic, analogous to the S4 diamond.

% Give a posite as a poset and a nucleus on its downsets, rather than a coverage. Equivalent, but expository.

% Every finite locale has a presentation as a DSC. Further, there is a correspondence of sets between DSCs (up to renaming) and posites equipped with the BL topology.
% Extension to an equivalence of categories by...


By analogy with other finitely generated objects, we give the following:

\begin{definition}
A \textbf{finitely generated posite} is a locale presented as a finite set \(S\) and a nucleus on its distributive lattice of downsets, \(\Oc(S)\). It is straightforward to check that every finite locale has a presentation as a finitely generated posite (and indeed multiple such presentations).
% TODO check.
\end{definition}

There is a morphism, \(\Lc\) that sends DSCs to finitely generated posites. It acts as identity on the set of all events, and generates a nucleus that acts as identity on elements of \(Pc(E)\) which are complete,  transforms the dependency relation into a nucleus by [....]

% Take the powerset of the event set, then impose the double-basis topology. Then impose the ordering on that.

\begin{theorem}
DSC and FinGenPosite are equivalent as categories
\end{theorem}

% TODO need a nucleus on the double powerset

%Finitely generated locales are locales given as a downset over a set and a topology.

%TODO maybe? its a conjecture

% Really its a free set with a coverage. Consider morphisms of sites



\subsection{Version Parameterizations of Dependency Structures and their Posets}

The mapping from an actual repository to a DCS is incomplete in that it throws away information about two packages being different "versions" of the same thing. Here we introduce a data structure that captures that additional information. A higher version of a package is, intuitively, something that is "almost the same, but better." Similarly, in a DSC, some events are in a sense "universally more powerful" than other events, in that they enable at least all the things enabled by the events they cover. We say an event is a "higher version" of another event if for possible dependency set of every event containing the lower event, there is also a possible dependency set for that event which differs only in that it contains the higher event instead.

\begin{definition}
A \textbf{version parameterization} of a DSC is an idempotent endofunction on events, from lower to higher, satisfying the above criteria; i.e. where for every possible dependency set of every event, there is another possible dependency set of that event where the lower versions have been substituted for higher versions. Idempotency here translates into the condition that no higher event is itself a lower event of something else.
\end{definition}

Versioning parameterizations on DSCs in turn give rise to related structures on their reachable dependency posets. Since a reachable dependency poset is complete under joins, for every two event sets differing only in versions of some events, we can take the node corresponding to the union of their event sets. This yields a idempotent monotone endofunction on nodes of the reachable dependency poset, sending ("contracting") the lower and higher event sets both to their corresponding union, which we term a \textbf{poset version parameterization} .

\begin{lemma}
Poset version parameterizations preserve existing meets and joins.
\end{lemma}
\begin{proof}
First we consider meets. If two sets had a meet before, and one is now "contracted" to a higher version, this may introduce new elements in the intersection only on the condition that these elements were also in the second set. But that would mean that the elements of the second set would also be contracted in the same way, as would the elements in the intersection itself.

Next, we consider joins. If two sets had a join before, and one was contracted to a higher version, then this would introduce new elements in the union. But those elements are the same element which are introduced by contracting the union directly.
\end{proof}

Since poset version parameterizations preserve existing meets and joins, they certainly preserve meets and maximal joins, and thus are acted on by \(\Mcc\).

\begin{theorem}
Given a DSC \((E,D)\) and a version parameterization with an induced poset version parameterization \(p\), then \(\Mcc(p)\) is a nucleus on \(\Mcc(E,D)\).
\end{theorem}
\begin{proof}
First we consider meet preservation. Since we know that \(p\) preserves meets, we only need observe that the action of \(\Mcc\) on morphisms preserves meet-preservation. This follows from the fact that \(\Mcc\) itself preserves meets.

%TODO check, spell out more.

Next, we consider contractivity. We already defined \(p\) to be contractive. So now we only need observe that \(\Mcc\) preserves contractivity. For this to fail, it would need to extend a morphism so that some element was mapped to something below itself. But since contractivity is preserved for the basis join-irreducible elements, it must be preserved for all elements.

Finally, we consider idempotence. Again, this is given by idempotence of \(p\) combined with \(\Mcc\) preserving idempotence of join-irreducible elements.
\end{proof}


\section{Free Distributive Lattices and the Logic(s) of Dependency Structures}
As discussed above, is part of the basic theory of Heyting algebras that they possess an internal intuitionistic logic. Here we sketch how it works in our particular case.

Given a DSC \((E,D)\), we construct a language consisting of atoms given by join-irreducible elements of the reachable dependency poset (which may be thought of, as above, as events subscripted with their dependency trace), completed under the standard logical connectives. Every formula in this language corresponds to a particular node in the Merkle-lattice of our DSC, \(\Mcc(E,D)\).  Conjunction corresponds to meet, disjunction to join, and implication to the relative pseudo-complement. This is a logic of reachable states of our system, and their traces, which describes all possible states of the system as disjunctions of join-irreducible states. Given two event sets, considered as reachable states, disjunction gives the set of events that have occurred in either state. Conjunction gives the set of events which have occurred in both states.

%TODO explain what relative pseudo-complement interprets as in this setting 

From the standpoint of nondeterministic concurrent semantics, this is perfectly reasonable. From two configurations (considered as nondeterministic traces), "or" gives us their combination, and "and" gives us the least trace which they share between them. However, from the standpoint of dependencies, this is insufficient. It is a logic of states in the system, but it is not a logic \textit{about} states in the system -- i.e., requirements. A a logical "or"  would express that we're asking for one or the other event set as such, unlike the disjunction given here. Similarly, a logical "and" would express that we are requiring all events in both sets (i.e. what disjunction actually provides). Without this missing logical "or", for example, we cannot give a formula that specifies "any way to reach an event, regardless of the dependencies," as in the logic thus far, \(a_b \bor a_c\) specifies \textit{both} ways, as opposed to \textit{either} way.

Some further mathematical constructions are required to build back up to the logic we'd really like. Returning to the finite characterization of the extended Bruns-Lakser completion, we note that taking the downsets of the join-irreducible elements is effectively taking their free join-completion. Dually, taking up-sets is a free meet-completion. In fact, for a discrete set \(S\), \(\Uc(\Oc(S))\) is the \textbf{free distributive lattice} over S, which is a well studied mathematical object [CITE]. The elements of this belong to \(\Pc(\Pc(S)\) and consist of its \textit{irredundant subsets}. This is to say that we can read these elements as logical expressions in disjunctive normal form, for example as \(ab \bor c\). An irredundant set is one in which the clauses have been simplified -- i.e. in which \(a \bor ab\) has been reduced to simply \(a\).  In this construction, join remains union -- but now it is not a union of sets of atoms, but a union of \textbf{sets of sets} of atoms. Meet however, is no longer intersection. Rather, it becomes convolution, just as in standard logic! This is to say that the meet (and) of \(a \bor b\) and \(c \bor d\) becomes the irredundant (simplified) core of \(a \band c \bor a \band d \bor b \band c \bor b \band d\).

There are some interesting open problems regarding free distributive lattices, in particular the search for a closed form expression that counts the elements of such a construction over a set of a given size (on which more later).

The free distributive lattice construction extends to any poset \(P\), where the up-sets of the downsets are the free distributive lattice of a poset. This is a less studied, but still known construction. As above, join is union and meet is convolution. But when we "multiply" two atoms, we don't simply conjoin them. Rather, we take their join in the underlying poset. Hence we arrive at a quotient of the free distributive lattice generated when the elements of \(P\) are considered simply as a set, and potentially a much smaller one. For example, if the underlying poset is a linear order, then the free distributive lattice over it is equivalent to the original poset. In general, the more ordering in the original poset, the smaller the resultant free distributive lattice over it.

 In line with this, we can build the free meet-completion of the idempotent distributive lattice completion of a dependency poset by the compound construction \(\Uc(\Oc(\Jc(P)))\). For a DSC, the internal logic of this construction, which we call \(\UMc(E,D)\) has precisely the same syntax as that of the internal logic \(\Mcc(E,D)\). However, while the normal form of the latter consisted of a set of irreducible elements, the normal form for our new construction consists of a set of sets -- i.e. a boolean expression of these elements given in disjunctive normal form, and simplified by reduction with regards to redundancy as well as the ordering relationship given on the join-irreducibles. We refer to the elements of \(\UMc(E,D)\) not as event sets, but as event equations, since they consist of events related by conjunction and disjunction. Further, it should be noted that since \(\Mcc\) is functorial, and taking up-sets is functorial, then the compound \(\UMc\) is also functorial, and has an action on morphisms of posets.
 
%TODO illustrations

%TODO explain what relative pseudo-complement interprets as in this setting 


As discussed above this nucleus equips the internal logic of \(\Mcc(E,D)\) with a modal operator. We can interpret this operator as "round (or upgrade) to the highest version," and give corresponding interpretations of the meaning of its basic laws. \(x \rightarrow \Dia{x}\) tells us that everywhere \(x\) is valid, so too is its highest version (which is precisely how we constructed our modality to begin with). \(\Dia\Dia{x} \rightarrow \Dia{x}\) tells us that the highest version of the highest version is just the highest version. \((x \rightarrow y) \rightarrow (\Dia{x} \rightarrow \Dia{y})\)  tells us that if an implication holds for an event set, then the highest version of that set implies the highest version of the consequent. Finally, \(\Dia(x \band y) = \Dia{x} \band \Dia{y}\) tells us that the highest version of a conjunction may be computed as the conjunction of the highest versions of its constituents.

In fact, the this modality is powerful enough to yield an internal "bind" operator. Since we have a strength that gives \(x \band \Dia{y} \rightarrow \Dia(x \band y)\), and since we have an internal evaluation \(x \band (x \rightarrow y) \rightarrow y\), then we have \(\Dia{x} \band (x \rightarrow \Dia{y}) \rightarrow \Dia{y}\), read as "if we know that some version of x implies the highest version of y, then to imply y it suffices to consider the highest version of x." Hence, we have a computational monad in the sense of Moggi [CITE], and rounding (or "upgrading") is an "effect".

% Show that this is not true for disjunction -- counterexample.

Since \(\Mcc(E,D)\) is really a logic of concurrent execution, it makes good sense to consider versioning parameterizations not only from the standpoint of dependencies, where they occur most naturally, but also from the standpoint of concurrent semantics. If we view reachable configurations as nondeterministic traces, then the "highest version" is most general state with regards to any given configuration -- i.e. associates to any trace another one which provides at least as many options for further execution as before, and possibly more. The various laws can be read similarly. In fact, .

To fully explore the dependency standpoint. we would also like to interpret the modal operator associated with a versioning parameterization in \(\UMc(E,D)\) as well. This brings us to the following:

\begin{theorem}
Given a DSC \((E,D)\) and a version parameterization with an induced poset version parameterization \(p\), then \(\UMc(p)\) is a nucleus on \(\UMc(E,D)\). Furthermore, it preserves not only meets, but also joins.
\end{theorem}
\begin{proof}
We have already established that \(\Mcc\) yields a contractive endofunction on a distributive lattice (and more). Since taking up-sets is functorial, we know that this in turn yields a contractive endo-function on the compound. Furthermore, since the up-set functor runs from all posets to distributive lattices, it lifts order-preserving morphisms of posets into meet and join preserving morphisms of distributive lattices. Hence, the induced contractive endofunction preserves both meets and joins, and is a nucleus.
\end{proof}

This in turn induces a modality, as above, but now on the logic not only of event sets, but of event equations. Thus we get a simple type theory of package dependencies with an internal "version policy" modality that obeys the usual algebraic identities, and lets us treat contraction along the version policy as a monadic effect. Package version policies have been a subject of some debate in the applied world. But as we now see, all told, a package version policy in a dependency structure is just a certain type of monad in the free distributive lattice over the join irreducibles of the associated poset.

% Free Distributive Lattices and Nuclei

\section{Solutions to Dependency Problems}
In future work, we will discuss a general theory of incompatibilities. For now, we define a general notion of a dependency problem.

\begin{definition}
A dependency problem in a DSC \((E,D)\) is the pair of a formula \(\phi\) in \(\UMc(E,D)\) and a monotone increasing objective function of type \(\Pc(E) \rightarrow \mathbb{R}\). A solution to such a problem is an event set which satisfies the formula and minimizes the objective function.
\end{definition}

Without further constraints on the objective function, then clearly there is no more efficient method of solving a dependency problem than brute force search over all event sets which satisfy the formula. But that does not mean there is nothing to be said. Rather, it directs us to focus on the efficiency of enumerating such sets, and also the sum total quantity of such sets.

\section{Related and Future Work}
% In fact, since \(\Dia\) commutes with meets (i.e. \(\Dia(x \band y) = \Dia{x} \band \Dia{y}\) and  \(\Dia(x \bor y) = \Dia{x} \bor \Dia{y}\))

\end{document}

\begin{comment}
Appendix matter
\textbf{Thm. (Birkhoff)}: When L is a finite distributive lattice, \(\Oc(\Jc(L)))\) is an equivalence, and for any finite poset P,  \(\Jc(\Oc(P)))\) is an equivalence. Further, this equivalence extends to a functorial equivalence between the categories \(FinPos\) and \(FinDLat\), with monotone functions on posets corresponding to homomorphisms of distributive lattices.

\end{comment}

% Nielsen, M., Plotkin, G., Winskel, G., Petri nets, Event structures and Domains, part 1. Theoretical Computer Science, vol. 13 (1981).
% heyting algebras: http://www2.math.uu.se/~palmgren/tillog/heyting3.pdf
% also A.S. Troelstra and D. van Dalen (1988).Constructivism in Mathematics, Vol. I &II.North-Holland.

% BL
% The Dedekind MacNeille site completion of a meet semilattice
% Richard N. Ball, Aleš Pultr, Joanne Walters Wayland

% Grothendieck topologies on posets∗ A.J. Lindenhovius


% Incompatibilities and knowledge representation
% Program to calculate dependency coverings of hackage.

% Cite SHAKE PAPER