\documentclass{beamer}
\usetheme{Warsaw}
\usecolortheme{spruce}
%\usepackage[parfill]{parskip}    		% Activate to begin paragraphs with an empty line rather than an indent
\usepackage{graphicx}				% Use pdf, png, jpg, or eps§ with pdflatex; use eps in DVI mode
								% TeX will automatically convert eps --> pdf in pdflatex
\usepackage{amssymb}
\usepackage{xypic}
\usepackage{cancel}
\usepackage{mathtools}
\newtheorem*{remark}{Remark}




%SetFonts

%SetFonts

%------------------------------------------------
%            Symbols in "mathcal"
%------------------------------------------------

\newcommand{\Mcc}{\mathcal{M}}
\newcommand{\UMc}{\mathcal{UM}}
\newcommand{\Wc}{\mathcal{W}}
\newcommand{\Oc}{\mathcal{O}}
\newcommand{\Pc}{\mathcal{P}}
\newcommand{\Ucc}{\mathcal{U}}
\newcommand{\Jc}{\mathcal{J}}
\newcommand{\Cc}{\mathcal{C}}
\newcommand{\Lcc}{\mathcal{L}}
\newcommand{\Dcc}{\mathcal{D}}
\newcommand{\BLc}{\mathcal{BL}}

\newcommand{\band}{\mathbin{\&}}
\newcommand{\bor}{\mathbin{|}}
\newcommand{\Dia}{\diamondsuit}

\newcommand{\Mf}{\mathfrak{m}}
\newcommand{\Sf}{\mathfrak{s}}
\newcommand{\Wf}{\mathfrak{w}}
\newcommand{\Hf}{\mathfrak{h}}

\DeclarePairedDelimiter\ceil{\lceil}{\rceil}
\DeclarePairedDelimiter\floor{\lfloor}{\rfloor}

\title{Topological Aspects of Dependency Structures}
\subtitle{CUNY Topology Seminar, October 22, 2019}
\author{Gershom Bazerman, Awake Security\linebreak(jww Raymond Puzio, Albert Einstein Institute)}

\begin{document}
\begin{frame}
		\titlepage
\end{frame}
\begin{frame}
\frametitle{Outline}
\tableofcontents
\end{frame}

\section{Dependency Structures}

\begin{frame}
\frametitle{Motivation}
\begin{itemize}
\item Package repositories
\item Concurrent semantics (Petri nets, CCS, CSP, \(\pi\)-calculus, etc.)
\item Knowledge representation (proof dependencies, course dependencies, chapter dependencies)
\end{itemize}
\begin{block}{Two key questions}
Two key questions: \textit{compositionality} (external, gros), and \textit{reachability} (internal, petit).
\end{block}
\end{frame}

\begin{frame}
\frametitle{Existing Order-Theoretic Models}
\begin{block}{General Event Structures (Neilsen, Plotkin, Winskel)}
Model dependency, conflict, choice. Hard to reason about!
\end{block}
\begin{block}{Event Structures (Winskel)}
Model dependency, conflict, textbf{not} choice. Good properties, form a domain! Correspond to safe Petri nets, CCS.
\end{block}
\begin{block}{pomsets (partially ordered multisets) (Pratt)}
Model dependency, textbf{not} conflict, textbf{not} choice. Compose beautifully. Relate to Kleene algebras.
\end{block}
\end{frame}

\begin{frame}
\frametitle{Our Approach}
\begin{block}{Dependency Structures with Choice (B., Puzio)}
Model dependency, textbf{not} conflict, choice. Nice properties. Relate to locales and constructive logic. Haven't studied composition.
\end{block}
\begin{block}{Dependency Structures with Choice and Conflict (B., Puzio)}
Model dependency, conflict, choice. Future work! Should allow us to relate GES to Directed Topology.
\end{block}
\end{frame}

\begin{frame}
\frametitle{Pre-DSCs}
\begin{definition}
A \textbf{Pre-Dependency Structure with Choice} is a pair \((E, D : E \rightarrow \Pc(\Pc(E)))\) where \(E\) is a finite set of events, and \(D\) is a non-nullary mapping from \(E\) to its double powerset, to be interpreted as mapping each event to a set of alternative dependency requirements -- i.e. to a predicate in disjunctive normal form ranging over variables drawn from \(E\).
\end{definition}
\end{frame}


\begin{frame}
\frametitle{DSCs}
\begin{definition}
A \textbf{Dependency Structure with Choice} (DSC) is a pre-DSC with \(D\) satisfying  appropriate conditions of transitive closure and cycle-freeness. We define \(X\) as a \textbf{possible dependency set} of \(e\) if \(X \in D(e)\). We call an event set \(X\) a \textbf{complete event sent} if for every element \(e\) there is a possible dependency set \(Y\) of \(e\) such that \(Y \subseteq X\). A pre-DSC is a DSC if every possible dependency set of every element is complete, and no possible dependency set of any element contains the element itself. Pre-DSCs may be completed into DSCs by a transitive closure operation.
\end{definition}
\end{frame}	

\begin{frame}
\frametitle{reachable dependency posets}
Observation: A DSC has an associated reachable dependency poset (an "unwinding" or "configuration family") which is a subposet of \(\Pc(E)\) generated by possible dependency sets augmented by their "parent" and ordered by inclusion. A rdp has all joins, but only some meets.

% Give canonical example.
\end{frame}

\begin{frame}
\frametitle{A Goal}
A rdp is \textit{almost} the frame of opens of a topological space. Our aim is to complete it into one so that we can analyze dependency structures by topological means.
% Give canonical example.
\end{frame}

\section{Distributive Lattices}

\begin{frame}
\frametitle{Definitions}
\begin{itemize}
\item A \textbf{Lattice} is a poset with all finite meets (greatest lower bounds) and joins (least upper bounds).

\item A \textbf{Distributive Lattice} is a lattice such that \(x \vee (y \wedge z) = (x \vee y) \wedge (x \vee z)\). 

\item finite distributive lattices are one to one with finite frames (finite meets, arbitrary joins, distribution), and hence finite sober spaces.

\item \textbf{\(\Jc(P)\)} is the subposet of the join-irreducible elements (including nullary joins) of \(P\).

\item \textbf{\(\Oc(P)\)} is the distributive lattice generated by the downsets of \(P\) under inclusion (sometimes known as the ideals).
\end{itemize}
\end{frame}	

\begin{frame}
\frametitle{Birkhoff}
\begin{theorem}
\textbf{Birkhoff}: When L is a finite distributive lattice, \(\Oc(\Jc(L)))\) is an equivalence, and for any finite poset P,  \(\Jc(\Oc(P)))\) is an equivalence. Further, this equivalence extends to a functorial equivalence between the categories \(FinPos\) and \(FinDLat\), with monotone functions on posets corresponding to homomorphisms of distributive lattices.
\end{theorem}

Observation: the Birkhoff completion of a reachable dependency poset is not very interesting, and, relatedly, not in the least idempotent.
\end{frame}


\begin{frame}
\frametitle{Bruns-Lakser}
An \textbf{admissible set} is a subset \(S\) of a meet-semilattice \(P\) in which for all \(x\), \(x \wedge \bigvee(S) = \bigvee(x \wedge S)\).

\begin{theorem}
(Bruns-Lakser, MacNeille, Gherke-Van Gool) The partially ordered set of all admissible sets of a meet-semilattice \(P\) is a distributive lattice, \(\BLc(P)\). There exists an injection \(bl : P \rightarrow \BLc(P)\), which preserves all meets and joins of admissible sets. Furthermore, any morphism to a distributive lattice that preserves all meets and joins of admissible sets, \(f : P \rightarrow D\), factors uniquely into the injection \(bl : P \rightarrow \BLc(P)\) followed by a distributive lattice homomorphism \(g : \BLc(P) \rightarrow D\). This construction is known as the distributive envelope, injective hull, or Bruns-Lakser completion.
\end{theorem}
\end{frame}

\begin{frame}
\frametitle{BL for posets, finite case}

\begin{theorem}
(B., Puzio) For a finite poset, the distributive envelope may be constructed as \(\Oc(\Jc(P)))\), with an injection that sends join-irreducible elements to their downsets, and composite elements to the union of their join-irreducible basis. When \(P\) is a meet semilattice, this coincides with the Bruns-Lakser completion.
\end{theorem}

Observation: Extended BL is idempotent (i.e. a genuine completion), and furthermore it is topologically and logically meaningful.
\end{frame}


	\section{Locales}
	\section{Versioning}
	\section{Free Distributive Lattices}
	
	\frame {
		\frametitle{Concurrency}
		\[\frac{-b \pm \sqrt{b^2 - c}}{2a}\]
	}
	\frame{
		\frametitle{Sample Page 2}
		\framesubtitle{An Example of Lists}
		\begin{itemize}
			\item 1
			\item 2
			\item 3
		\end{itemize}
	}
	\frame{
	    \frametitle{Paragraph Content}
	    This is a paragraph.
	}
\end{document}
