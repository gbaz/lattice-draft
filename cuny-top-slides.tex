\documentclass{beamer}
\usetheme{Warsaw}
\usecolortheme{spruce}
%\usepackage[parfill]{parskip}    		% Activate to begin paragraphs with an empty line rather than an indent
\usepackage{graphicx}				% Use pdf, png, jpg, or eps§ with pdflatex; use eps in DVI mode
								% TeX will automatically convert eps --> pdf in pdflatex
\usepackage{amssymb}
\usepackage{xypic}
\usepackage{cancel}
\usepackage{mathtools}
\usepackage{tikz-cd}
\newtheorem*{remark}{Remark}


%SetFonts

%SetFonts

%------------------------------------------------
%            Symbols in "mathcal"
%------------------------------------------------

\newcommand{\Mcc}{\mathcal{M}}
\newcommand{\UMc}{\mathcal{UM}}
\newcommand{\Wc}{\mathcal{W}}
\newcommand{\Oc}{\mathcal{O}}
\newcommand{\Pc}{\mathcal{P}}
\newcommand{\Ucc}{\mathcal{U}}
\newcommand{\Jc}{\mathcal{J}}
\newcommand{\Cc}{\mathcal{C}}
\newcommand{\Lcc}{\mathcal{L}}
\newcommand{\Dcc}{\mathcal{D}}
\newcommand{\BLc}{\mathcal{BL}}

\newcommand{\band}{\mathbin{\&}}
\newcommand{\bor}{\mathbin{|}}
\newcommand{\Dia}{\diamondsuit}

\newcommand{\Mf}{\mathfrak{m}}
\newcommand{\Sf}{\mathfrak{s}}
\newcommand{\Wf}{\mathfrak{w}}
\newcommand{\Hf}{\mathfrak{h}}

\DeclarePairedDelimiter\ceil{\lceil}{\rceil}
\DeclarePairedDelimiter\floor{\lfloor}{\rfloor}

\title{Topological Aspects of Dependency Structures}
\subtitle{CUNY Topology Seminar, October 22, 2019}
\author{Gershom Bazerman}
\date{}
\institute{Awake Security}
\date[\today]{ \tiny{jww Raymond Puzio, Albert Einstein Institute}}


\begin{document}
\begin{frame}
		\titlepage
\end{frame}
\begin{frame}
\frametitle{Outline}
\tableofcontents
\end{frame}

\section{Dependency Structures}

\begin{frame}
\frametitle{Motivation}
\begin{itemize}
\item Package repositories
\item Concurrent semantics (Petri nets, CCS, CSP, \(\pi\)-calculus, etc.)
\item Knowledge representation (proof dependencies, course dependencies, chapter dependencies)
\end{itemize}
\begin{block}{Two key questions}
Two key questions: \textit{compositionality} (external, gros), and \textit{reachability} (internal, petit).
\end{block}
\end{frame}

\begin{frame}
\frametitle{Existing Order-Theoretic Models}
\begin{block}{General Event Structures (Neilsen, Plotkin, Winskel)}
Model dependency, conflict, choice. Hard to reason about!
\end{block}
\begin{block}{Event Structures (Winskel)}
Model dependency, conflict, \textbf{not} choice. Good properties, form a domain! Correspond to safe Petri nets, CCS.
\end{block}
\begin{block}{pomsets (partially ordered multisets) (Pratt)}
Model dependency, \textbf{not} conflict, \textbf{not} choice. Compose beautifully. Relate to Kleene algebras.
\end{block}
\end{frame}

\begin{frame}
\frametitle{Our Approach}
\begin{block}{Dependency Structures with Choice (B., Puzio)}
Model dependency, \textbf{not} conflict, choice. Nice properties. Relate to locales and constructive logic. Haven't studied composition.
\end{block}
\begin{block}{Dependency Structures with Choice and Conflict (B., Puzio)}
Model dependency, conflict, choice. Future work! Should allow us to relate GES to Directed Topology.
\end{block}
\end{frame}

\begin{frame}
\frametitle{Pre-DSCs}
\begin{definition}
A \textbf{Pre-Dependency Structure with Choice} is a pair \((E, D : E \rightarrow \Pc(\Pc(E)))\) \end{definition}
\begin{itemize}
\item \(E\) is thought of as a finite set of events.
\item \(D\) is thought of as mapping each event to a set of alternative dependency requirements -- i.e. to a predicate in disjunctive normal form.
\end{itemize}
\end{frame}


\begin{frame}
\frametitle{DSCs}
\begin{definition}
A \textbf{Dependency Structure with Choice} (DSC) is a pre-DSC with \(D\) satisfying appropriate conditions of transitive closure and cycle-freeness. 
\end{definition}
\(X\) is a \textbf{possible dependency set} of \(e\) if \(X \in D(e)\). An event set \(X\) is a \textbf{complete event sent} if for every element \(e\) there is a possible dependency set \(Y\) of \(e\) such that \(Y \subseteq X\). A pre-DSC is a DSC if every possible dependency set of every element is complete and cycle-free.
\end{frame}	

\begin{frame}
\frametitle{reachable dependency posets}
Observation: A DSC has an associated reachable dependency poset (an "unwinding" or "configuration family") which is a subposet of \(\Pc(E)\) generated by possible dependency sets augmented by their "parent" and ordered by inclusion. A rdp has all joins, but only some meets.

% Give canonical example.
\end{frame}

\begin{frame}
\frametitle{A Goal}
A rdp is \textit{almost} the frame of opens of a topological space. Our aim is to complete it into one so that we can analyze dependency structures by topological means.
% Give canonical example.
\end{frame}

\section{Distributive Lattices}

\begin{frame}
\frametitle{Definitions}
\begin{itemize}
\item A \textbf{Lattice} is a poset with all finite meets (greatest lower bounds) and joins (least upper bounds).

\item A \textbf{Distributive Lattice} is a lattice such that \(x \vee (y \wedge z) = (x \vee y) \wedge (x \vee z)\). 

\item finite distributive lattices are one to one with finite frames (finite meets, arbitrary joins, distribution), and hence finite sober spaces.

\item \textbf{\(\Jc(P)\)} is the subposet of the join-irreducible elements (including nullary joins) of \(P\).

\item \textbf{\(\Oc(P)\)} is the distributive lattice generated by the downsets of \(P\) under inclusion (sometimes known as the ideals).
\end{itemize}
\end{frame}	

\begin{frame}
\frametitle{Birkhoff}
\begin{theorem}
\textbf{Birkhoff}: When L is a finite distributive lattice, \(\Oc(\Jc(L)))\) is an equivalence, and for any finite poset P,  \(\Jc(\Oc(P)))\) is an equivalence. Further, this equivalence extends to a functorial equivalence between the categories \(FinPos\) and \(FinDLat\), with monotone functions on posets corresponding to homomorphisms of distributive lattices.
\end{theorem}

Observation: the Birkhoff completion of a reachable dependency poset is not very interesting, and, relatedly, not in the least idempotent.
\end{frame}


\begin{frame}[fragile]
\frametitle{Bruns-Lakser}
An \textbf{admissible set} is a subset \(S\) of a meet-semilattice \(P\) in which for all \(x\), \(x \wedge \bigvee(S) = \bigvee(x \wedge S)\).

\begin{theorem}
(Bruns-Lakser, MacNeille, Gherke-Van Gool) The partially ordered set of all admissible sets of a meet-semilattice \(P\) is a distributive lattice, \(\BLc(P)\). Furthermore, when \(D\) is distributive, and \(f\) preserves joins of admissible sets and all meets:

\begin{equation*}
\begin{tikzcd}
\BLc(P) \arrow[r, "!",dashed]            & D \\
P \arrow[u, hook] \arrow[ru, "f"'] &  
\end{tikzcd}
\end{equation*}
\end{theorem}

\end{frame}

\begin{frame}[fragile]
\frametitle{BL for posets, finite case}

\begin{theorem}
(B., Puzio) For a finite poset, the injective hull may be constructed as \(\Oc(\Jc(P)))\), with an injection that sends join-irreducible elements to their downsets, and composite elements to the union of their join-irreducible basis. Furthermore, 

\begin{equation*}
\begin{tikzcd}
\BLc(P) \arrow[r, "!",dashed]            & D \\
P \arrow[u, hook] \arrow[ru, "f"'] &  
\end{tikzcd}
\end{equation*}

\end{theorem}

Corollary: (Finite) distributive lattices are a reflective subcategory of (finite) posets with "distributive" morphisms.
\end{frame}

\begin{frame}
\frametitle{Extended BL is idempotent, and meaningful}
\begin{figure}
\begin{minipage}[c]{0.4\textwidth}
\begin{equation*}
    \xymatrix{& abc \ar@{-}[dl] \ar@{-}[d] \ar@{-}[dr] & \\
      \underline{ab} \ar@{-}[d] \ar@{--}[dr] & bc \ar@{-}[dl] \ar@{-}[dr] &
        \underline{ac} \ar@{-}[d] \ar@{--}[dl] \\
        \underline{b} & \xcancel{a}  & \underline{c} }         
%      b \ar@{-}[dr] & \xcancel{a} \ar@{--}[d] & c \ar@{-}[dl] \\
%       & \emptyset &}
\end{equation*}
\end{minipage}
\begin{minipage}[c]{0.08\textwidth}
  \begin{equation*}
    \xymatrix{\ar@2{~>}[r]^{\Mcc} &}
  \end{equation*}
\end{minipage}
\begin{minipage}[c]{0.4\textwidth}
\begin{equation*}
    \xymatrix{& a_ba_cbc \ar@{-}[dl] \ar@{-}[dd] \ar@{-}[dr] & \\
      \mathbf{a_bbc} \ar@{-}[d] \ar@{-}[dr] & &
        \mathbf{a_cbc} \ar@{-}[d] \ar@{-}[dl]\\
        \underline{a_bb} \ar@{-}[d] & bc \ar@{-}[dl] \ar@{-}[dd] \ar@{-}[dr] &
          \underline{a_cc} \ar@{-}[d] \\
         \underline{b} \ar@{-}[dr] & & \underline{c} \ar@{-}[dl] \\
      & \mathbf{\emptyset} }
\end{equation*}
\end{minipage}
\end{figure}
\end{frame}

\section{Locales}
\begin{frame}
\frametitle{Definitions}
A \textbf{Heyting algebra} is a lattice with a unique top and bottom element, and a special "implication" operation called the \textbf{relative pseudo-complement} (\(a \rightarrow b\)) which yields the unique greatest element \(x\) such that \(a \wedge x \le b\). A \textbf{complete Heyting algebra} is a Heyting algebra such that it is also a complete lattice. Finite distributive lattices are one-to-one with finite Heyting algebras.

A \textbf{frame} is a complete Heyting algebra, considered in a category where morphisms preserve finite meets and arbitrary joins. A \textbf{locale} is a frame, but in a category with morphism reversed. \(FinLoc = FinFrm^{op} = FinDLat^{op}\)
\end{frame}

\begin{frame}
A \textbf{nucleus} is the algebraic structure on a frame that gives a sublocale. It given as a monotone function on a frame \(j : L \rightarrow L\), satisfying three properties. First: \(j(a \wedge b) = j(a) \wedge j(b)\). Second, \(a \le j(a)\). Finally, \(j(ja) \le j(a)\). These may be summarized as saying that it is (finite) meet-preserving, contractive (in this case, inflationary) and idempotent. An element of \(L\) is said to be \(j\)-closed if \(j(a)=a\). Further, if \(L\) is a frame, then \(L/j\), which consists only of \(j\)-closed elements, is also a frame, and there exists a surjective frame homomorphism \(j^* : L \rightarrow L/j\). 

If we view a frame as a category, a nucleus is just a left-exact idempotent monad; if we view a frame as a decategorified topos, a nucleus is a topology; and if we view a frame as generated by its internal logic, a nucleus gives a "possibility" or "locally true" modal operator on that logic, analogous to the S4 diamond.
\end{frame}

\begin{frame}
\frametitle{locales in a nutshell}
Frames correspond to the "frame of opens" of a topological space. Frames/Locales are "generalized spaces" which may have "too many points, or not enough". Frames "with enough points" are 1-1 with sober spaces. All spaces may be "soberified" as a reflector.  Localic homology and homotopy is relatively underdeveloped. By their connection to Heyting algebras, frames are complete with regards to intuitionistic logic. (A statement is provable in IL iff it is true in all frames). Importantly, they let us relate topologies and logical modalities.
\end{frame}

\begin{frame}
\frametitle{The BL-topology}
\begin{definition}
The \textbf{Bruns-Lakser topology} on a finite locale is a function generated by the following stepwise procedure. First we consider the join-irreducible elements of the underlying poset of the locale (i.e. the poset of its join-irreducible elements). We term such elements the "double basis" of the locale. For double-basis elements, we define \(bl\) as the identity. For all joins of such elements, \(bl\) is again the identity. And for all meets of all elements thus far enumerated, \(bl\) is again the identity. For all other elements, \(bl\) is the meet of all idempotent elements greater than or equal to it. Since we have explicitly added all necessary meets, the result is necessarily an idempotent.
\end{definition}

\end{frame}

\begin{frame}
\frametitle{Localic presentations of DSCs}

\begin{lemma}
The Bruns-Lakser topology a nucleus and further is the least nucelus which preserves the join-irreducible elements of the underlying poset of a finite frame.
\end{lemma}

\begin{theorem}
The set of DSCs is one-to-one with the set of finite locales, equipped with the Bruns-Lakser topology.
\end{theorem}

Question: What morphisms do we need to define to extend this to an equivalence of categories?

\end{frame}

\section{Versioning}
\begin{frame}
\begin{definition}
A \textbf{version parameterization} of a DSC is an idempotent endofunction on events, from lower to higher, where for every possible dependency set of every event, there is another possible dependency set of that event where the lower versions have been substituted for higher versions. Idempotency here translates into the condition that no higher event is itself a lower event of something else.

A \textbf{poset version parameterization} is the induced endofunction on a reachable dependency poset generated by a version parameterization on a DSC
\end{definition}


\end{frame}


\begin{frame}

\begin{lemma}
A \textbf{ponucleus} is an idempotent monotone endofunction on a poset which preserves existing finite meets.
Every poset version parameterization is a ponucleus that in addition preserves existing joins.
\end{lemma}

\begin{theorem}
Given a join-preserving ponucleus \(j\), on a poset \(P\),  \(\Mcc(j)\) is a nucleus on \(\Mcc(P)\)
\end{theorem}

Corollary: given a DSC \((E,D)\) and a version parameterization with an induced poset version parameterization \(p\), then \(\Mcc(p)\) is a nucleus on \(\Mcc(E,D)\).
\end{frame}

\section{Free Distributive Lattices}
\begin{frame}
\frametitle{Motivation}
"Internal logic" of the locale of a DSC  -- i.e., \(\Mcc(E,D)\)
\begin{itemize}
\item Atoms = "events with traces"
\item Objects = traced event sets
\item \(\Dia\) =  "round the version up"
\item \(\&\) = intersection
\item \(||\) = union
\end{itemize}

Not the logic we want! (It is a logic "of" states rather than "about" states).
\end{frame}

\begin{frame}
\begin{itemize}
\item A set \(S\) is \textit{redundant} with regards to \(T\) if \(T \subset S\). 
\item The irredundency quotient is a nucleus, and the double powerset lattice, qotiented by irredundancy, is the free distributive lattice. 
\item This gives positive formulae in disjunctive normal form, quotiented by all laws of first order intuitionistic logic.
\item This is equivalent to the lattice \(\Ucc(\Oc(S))\) (upsets of downsets).
\item This construction extends from sets to posets.
\item (Thm., B. Puzio) This construction sends a join-preserving ponucleus to a nucleus.
\end{itemize}
\end{frame}

\begin{frame}
\frametitle{The Free Distributive Lattice of a DSC}
The (modal) internal logic the free locale of a DSC -- i.e. \(\UMc(E,D)  = \Ucc(\Oc(\Jc(rdp(DSC))))\)

\begin{itemize}
\item Atoms = "events with traces"
\item Objects = \textbf{Sets of} traced event sets (as formulae in dnf)
\item \(\Dia\) =  "round the version up"
\item \(\&\) = actual logical conjunction
\item \(||\) = actual logical disjunction
\end{itemize}

This \textbf{is} the logic you're looking for.

\end{frame}

\section{Solutions to Dependency Problems}
\begin{frame}
\begin{definition}
A dependency problem in a DSC \((E,D)\) is the pair of a formula \(\phi\) in \(\UMc(E,D)\) and a monotone increasing (i.e. growing as further elements are added to the source set) objective function of type \(\Pc(E) \rightarrow \mathbb{R}\). A solution to such a problem is an event set which satisfies the formula and minimizes the objective function.
\end{definition}

Observation: Detecting \textit{compatible} event sets can be formulated as a dependency problem.
\end{frame}

\begin{frame}
The cost of solving a dependency problem over a poset is bounded by the maximum number of disjuncts in a normalized dnf formula over that poset. This is the same as the maximal antichain in the downsets of that poset -- i.e. the \textbf{width}.

\begin{theorem}
\textbf{(Sperner, 1928)}: The powerset of a set with \(n\) elements, under inclusion ordering, has a width of \(\binom{n}{\ceil{n/2}}\).
\end{theorem}
\end{frame}

\begin{frame}
\begin{theorem}
(B., Puzio): Define \(\Mf(a,b)\) as the central (maximal) coefficient of the formal polynomial expansion of \((1 + x + x^2 ... + x^a)^b\). Define \(\Hf(P)\) as the height of a poset, i.e. the length of its longest chain. Given any poset \(P\), then:
\begin{equation*}
 \Wf(\Oc(P)) \le \Mf(2,\Wf(P)) * \Mf(\Hf(P),\ceil{\Wf(P)/2})
\end{equation*}
\end{theorem}
\end{frame}

\begin{frame}
\frametitle{Future Work}
Mathematical:
\begin{itemize}
\item Develop a homology theory over conflicts using this framework. It should look like discrete morse theory.
\item Relate this construction to directed topology.
\end{itemize}

Applied:
\begin{itemize}
\item Characterize the difficulty of problems for SAT solvers (and possibly give a general construction of "eager" SMT solving).
\item Develop a modal language of dependencies for software packaging (Haskell, Nix, etc.).
\item Provide a sound basis for the organization of data in Hales' Formal Abstracts project. 
\end{itemize}

\end{frame}

\end{document}
